\chapter[QoS]{Quality-of-Service}

\gls{qos} is used to guarantie a minimum of service level to select applications. Often this encompasses \gls{voip} applications and \gls{av} applications being allocated the highest priority. It\tsq{s} not uncommon to allocate \gls{nmt} to the high priority queue, too.

Different mechanisms of handling access to network ressorces is used.
\begin{itemize}
    \item \itemhead{\gls{nac}}
    \begin{enumerate}
        \item Which applications has access to what level of network ressources.
    \end{enumerate}
    \item \itemhead{Traffic Control}
    \begin{enumerate}
        \item Scheduling of traffic,
        \item classifiying traffic,
        \item marking packets based upon priority,
        \item marking packets based upon shaping traffic.
    \end{enumerate}
\end{itemize}

\section{Concepts}

\gls{qos}\tsq{s} goal is a differentiated prioritazion of packets parsing thorugh the network based upon the following concepts:
\begin{enumerate}
    \item Bandwidth,
    \item latency,
    \item jitter\footnote{Latency Variation}
    \item realiability\footnote{Pct. of packets discarded by any a router}.
\end{enumerate}

\section[Congestion Mgmt]{Congestion Management}

There are different ways to do congestion management. Which is in it\tsq{s} essence sorting of packets when a link reaches full capacity usage in the outgoing direction.

\begin{enumerate}
    \item \gls{fifo}: Classic store-and-forward. Oftentimes the default algorithm.
    \item \gls{pq}: Made to give stict priority to important traffic at each point \gls{pq} in the network where \gls{pq} is used.
    \item \gls{cq}: 
    \item \gls{wfq}: Traffic is diveded into flow based upon charactaristics \begin{mylist} \item \gls{dst} address, \item \gls{src} address, \item protocol, \item port number, \item socket. \end{mylist}. Flows is then allocated a part of the bandwidth relative to the number of ongoing conversations/flows.
    \begin{enumerate}
        \item Flow-based \gls{wfq},
        \item \gls{cbwfq},
        \item \gls{dwfq}.
    \end{enumerate}
    \item distributed class-based \gls{wfq}
    \item \gls{ip} \gls{rtp} priority
    \item \gls{llq}
    \begin{enumerate}
        \item \gls{llq}
        \item Distributed \gls{llq}: 
    \end{enumerate}
\end{enumerate}
