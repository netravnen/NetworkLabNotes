% Latex Packages
\usepackage{import}
\usepackage[T1]{fontenc}
\usepackage[utf8]{inputenc}
\usepackage{lmodern}
\usepackage{geometry}
\usepackage{listings}
\usepackage{color}
\usepackage[usenames,dvipsnames]{xcolor}
\usepackage{graphicx}
\usepackage[numbers,square,sectionbib,comma,nonamebreak,elide]{natbib}
\usepackage{float}
\usepackage[english]{babel}
\usepackage{fancyhdr}
\usepackage{wrapfig}
\usepackage{array}
\usepackage{lipsum}
\usepackage{fancybox}
\usepackage{varwidth}
\usepackage{enumitem}
\usepackage{titlepic}
\usepackage[nottoc]{tocbibind}
\usepackage{url}
\usepackage{babel}
\usepackage[showisoZ]{datetime2}
\usepackage{lastpage}


\bibliographystyle{unsrtnat} %styles list https://www.sharelatex.com/learn/Natbib_bibliography_styles


% Path where images are located relative
% to the file main.tex
\graphicspath{{img/}{figures/}}


% In which order to look after images in
% declared graphicspath{}'s
% 1. Low-quality JPG
% 2. Med-quality PNG
% 3. High-quality PDF
\DeclareGraphicsExtensions{.jpg,.png,.pdf}


\fancypagestyle{fancybook}{%
	\fancyhf{}%
	% Note the ## here. It's required because \fancypagestyle is making a macro (\ps@fancybook).
	% If we just wrote #1, TeX would think that it's the argument to \ps@fancybook, but
	% \ps@fancybook doesn't take any arguments, so TeX would complain with an error message.
	% You are not expected to understand this.
	\renewcommand*{\sectionmark}[1]{ \markright{\thesection\ ##1} }%
	\renewcommand*{\chaptermark}[1]{ \markboth{\chaptername\ \thechapter: ##1}{} }%
	% Increase the length of the header such that the folios 
	% (typography jargon for page numbers) move into the margin
	\fancyhfoffset[LE]{6mm}% slightly less than 0.25in
	\fancyhfoffset[RO]{6mm}%
	% Put some space and a vertical bar between the folio and the rest of the header
	\fancyhead[LE]{\color{GreenYellow}\thepage\hskip3mm\vrule\hskip3mm\leftmark}%
	\fancyhead[RO]{\color{GreenYellow}\rightmark\hskip3mm\vrule\hskip3mm\thepage}%
}
\pagestyle{fancybook}


% Use the roman numeric system for pagenumbers
\pagenumbering{roman}


\subimport{.}{commands} % Import user-defined commands


\definecolor{codegreen}{rgb}{0,0.6,0}
\definecolor{codegray}{rgb}{0.5,0.5,0.5}
\definecolor{codepurple}{rgb}{0.58,0,0.82}
\definecolor{backcolour}{rgb}{0.95,0.95,0.92}

%%%%%%%%%%%%%%%%%%%%%%%%%%%%%%%%%%%%%%%%%%%%%%%%%%%%%%%%%%%%%%%%%%%%%%%
%                                                                     %
% +--------------------------------------+                            %
% | The following colours is available   |                            %
% | with \usepackage[dvipsnames]{xcolor} |                            %
% +--------------------------------------+-------------------------------+
% | http://www.maths.adelaide.edu.au/anthony.roberts/LaTeX/ltxusecol.php |
% +----------------------------------------------------------------------+
% Apricot                                                             %
% Aquamarine                                                          %
% Bittersweet                                                         %
% Black                                                               %
% Blue                                                                %
% BlueGreen                                                           %
% BlueViolet                                                          %
% BrickRed                                                            %
% Brown                                                               %
% BurntOrange                                                         %
% CadetBlue                                                           %
% CarnationPink                                                       %
% Cerulean                                                            %
% CornflowerBlue                                                      %
% Cyan                                                                %
% Dandelion                                                           %
% DarkOrchid                                                          %
% Emerald                                                             %
% ForestGreen                                                         %
% Fuchsia                                                             %
% Goldenrod                                                           %
% Gray                                                                %
% Green                                                               %
% GreenYellow                                                         %
% JungleGreen                                                         %
% Lavender                                                            %
% LimeGreen                                                           %
% Magenta                                                             %
% Mahogany                                                            %
% Maroon                                                              %
% Melon                                                               %
% MidnightBlue                                                        %
% Mulberry                                                            %
% NavyBlue                                                            %
% OliveGreen                                                          %
% Orange                                                              %
% OrangeRed                                                           %
% Orchid                                                              %
% Peach                                                               %
% Periwinkle                                                          %
% PineGreen                                                           %
% Plum                                                                %
% ProcessBlue                                                         %
% Purple                                                              %
% RawSienna                                                           %
% Red                                                                 %
% RedOrange                                                           %
% RedViolet                                                           %
% Rhodamine                                                           %
% RoyalBlue                                                           %
% RoyalPurple                                                         %
% RubineRed                                                           %
% Salmon                                                              %
% SeaGreen                                                            %
% Sepia                                                               %
% SkyBlue                                                             %
% SpringGreen                                                         %
% Tan                                                                 %
% TealBlue                                                            %
% Thistle                                                             %
% Turquoise                                                           %
% Violet                                                              %
% VioletRed                                                           %
% White                                                               %
% WildStrawberry                                                      %
% Yellow                                                              %
% YellowGreen                                                         %
% YellowOrange                                                        %
%                                                                     %
%%%%%%%%%%%%%%%%%%%%%%%%%%%%%%%%%%%%%%%%%%%%%%%%%%%%%%%%%%%%%%%%%%%%%%% %user-defined colors


\lstdefinestyle{mystyle}{
    backgroundcolor=\color{backcolour},   
    commentstyle=\color{codegreen},
    keywordstyle=\color{magenta},
    numberstyle=\tiny\color{codegray},
    stringstyle=\color{codepurple},
    basicstyle=\footnotesize,
    breakatwhitespace=false,         
    breaklines=true,                 
    captionpos=b,                    
    keepspaces=true,                 
    numbers=left,                    
    numbersep=5pt,                  
    showspaces=false,                
    showstringspaces=false,
    showtabs=false,                  
    tabsize=4
}


\geometry{a4paper,margin=2cm}
\setlength{\columnsep}{1.5cm} %space between columns
\setlength{\headheight}{15pt}
\setlength{\footnotesep}{0.5cm} %space between footnotes:
\setlength{\skip\footins}{2cm} %space between the text body and the footnotes
\setlist[itemize,1]{leftmargin=\dimexpr 26pt-.2cm}
\setlist[itemize,2]{leftmargin=\dimexpr 26pt-.3cm}
\lstset{style=mystyle} %apply lst styling