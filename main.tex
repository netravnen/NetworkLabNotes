% Declare Document Class
\documentclass{article}

% Latex Packages
\usepackage[T1]{fontenc}
\usepackage[utf8]{inputenc}
\usepackage{lmodern}
\usepackage{geometry}
\usepackage{minted}
\usepackage{graphicx}
\usepackage{float}


% Path where images are located relative
% to the file main.tex
\graphicspath{{img/}}

\newcommand{\insertoutput}[1]{
    \begin{center}
        \includegraphics[width=12cm]{#1}
    \end{center}
}

% In which order to look after images in
% declared graphicspath{}'s
% 1. Low-quality JPG
% 2. Med-quality PNG
% 3. High-quality PDF
\DeclareGraphicsExtensions{.jpg,.png,.pdf}

% Package Params
\geometry{a4paper,margin=2cm}
% Used to NOT show red boxes around colons
% etc. when using the minted{} package.
\AtBeginEnvironment{minted}{\renewcommand{\fcolorbox}[4][]{#4}}

%
% BEGIN DOCUMENT
%
\begin{document}

% Which info to insert on the title page
\title{r17dinh409 Cisco Lab}
\author{Christoffer Hansen <zbcchhan11 at zbc.dk>}
\date{May 22 - June 30, 2017}
% Make title page contents
\maketitle

% Page break before starting Table Of Contents
\newpage
% Table Of Contents
\tableofcontents


%
% BEGIN SWITCH CONFIG
%
\section{Lab cfg}

% <!-- ROUTER -->

\subsection{Router}
\subsubsection{File: base.cfg}
\inputminted[frame=lines,framesep=2mm,baselinestretch=1.2,bgcolor=lightgray,fontsize=\footnotesize,linenos,breaklines=true]{powershell}{code/router/base.cfg}
\subsubsection{File: reset.tcl}
\inputminted[frame=lines,framesep=2mm,baselinestretch=1.2,bgcolor=lightgray,fontsize=\footnotesize,linenos,breaklines=true]{powershell}{code/router/reset.tcl}

% <!-- SWITCH -->

\subsection{Switch}
\subsubsection{File: base.cfg}
\inputminted[frame=lines,framesep=2mm,baselinestretch=1.2,bgcolor=lightgray,fontsize=\footnotesize,linenos,breaklines=true]{powershell}{code/switch/base.cfg}
\subsubsection{File: reset.tcl}
\inputminted[frame=lines,framesep=2mm,baselinestretch=1.2,bgcolor=lightgray,fontsize=\footnotesize,linenos,breaklines=true]{powershell}{code/switch/reset.tcl}

%
% END SWITCH CONFIG
%




%
% END DOCUMENT
%
\end{document}
