% Declare Document Class
\documentclass{article}

% Latex Packages
\usepackage[T1]{fontenc}
\usepackage[utf8]{inputenc}
\usepackage{lmodern}
\usepackage{geometry}
\usepackage{minted}
\usepackage{graphicx}
\usepackage{float}


% Path where images are located relative
% to the file main.tex
\graphicspath{{img/}}

% Custom commands
\newcommand{\insertoutput}[1]{
    \begin{center}
        \includegraphics[width=12cm]{#1}
    \end{center}
}
\newcommand{\pic}[2]{
    \begin{center}
        \transparent{0.4}
        \includegraphics[height=#1]{#2}
    \end{center}
}


% In which order to look after images in
% declared graphicspath{}'s
% 1. Low-quality JPG
% 2. Med-quality PNG
% 3. High-quality PDF
\DeclareGraphicsExtensions{.jpg,.png,.pdf}

% Package Params
\geometry{a4paper,margin=2cm}
% Used to NOT show red boxes around colons
% etc. when using the minted{} package.
\AtBeginEnvironment{minted}{\renewcommand{\fcolorbox}[4][]{#4}}

%
% BEGIN DOCUMENT
%
\begin{document}

% Which info to insert on the title page
\title{r17dinh409 Cisco Lab}
\author{Christoffer Hansen <zbcchhan11 at zbc.dk>}
\date{May 22 - June 30, 2017}
% Make title page contents
\maketitle

% Page break before starting Table Of Contents
\newpage
% Table Of Contents
\tableofcontents


%
% BEGIN SWITCH CONFIG
%
\section{Lab cfg}

% <!-- ROUTER -->

\subsection{Router}

% <!-- ROUTER: base.cfg -->

\subsubsection{File: base.cfg}

\begin{minted}[frame=lines,framesep=2mm,baselinestretch=1.2,bgcolor=lightgray,fontsize=\footnotesize,linenos,breaklines=true]{tcl}
tclsh
puts [ open "flash:base.cfg" w+ ] {
hostname __HOSTNAME__
ip domain-name cisco.tld
no ip domain lookup
interface range f0/1-24 , g0/1-2
shutdown
exit
vtp mode transparent
line con 0
no exec-timeout
logging synchronous
exit
end
}
tclquit
\end{minted}

% <!-- ROUTER: reset.tcl -->

\subsubsection{File: reset.tcl}

\begin{minted}[frame=lines,framesep=2mm,baselinestretch=1.2,bgcolor=lightgray,fontsize=\footnotesize,linenos,breaklines=true]{tcl}
tclsh
puts [ open "flash:reset.tcl" w+ ] {
typeahead "\n"
copy running-config startup-config
typeahead "\n"
erase startup-config
delete /force vlan.dat
delete /force multiple-fs
ios_config "sdm prefer dual-ipv4-and-ipv6 routing"
typeahead "\n"
puts "Reloading the switch in 1 minute, type reload cancel to halt"
typeahead "\n"
reload in 1 RESET.TCL SCRIPT RUN
}
tclquit
\end{minted}

% <!-- SWITCH -->

\subsection{Switch}

% <!-- SWITCH: base.cfg -->

\subsubsection{File: base.cfg}

\begin{minted}[frame=lines,framesep=2mm,baselinestretch=1.2,bgcolor=lightgray,fontsize=\footnotesize,linenos,breaklines=true]{tcl}
tclsh
puts [ open "flash:base.cfg" w+ ] {
hostname __HOSTNAME__
ip domain-name cisco.tld
no ip domain lookup
interface range f0/1-24 , g0/1-2
shutdown
exit
vtp mode transparent
line con 0
no exec-timeout
logging synchronous
exit
end
}
tclquit
\end{minted}

% <!-- SWITCH: reset.tcl -->

\subsubsection{File: reset.tcl}

\begin{minted}[frame=lines,framesep=2mm,baselinestretch=1.2,bgcolor=lightgray,fontsize=\footnotesize,linenos,breaklines=true]{tcl}
tclsh
puts [ open "flash:reset.tcl" w+ ] {
typeahead "\n"
copy running-config startup-config
typeahead "\n"
erase startup-config
delete /force vlan.dat
delete /force multiple-fs
ios_config "sdm prefer lanbase-routing"
typeahead "\n"
puts "Reloading the switch in 1 minute, type reload cancel to halt"
typeahead "\n"
reload in 1 RESET.TCL SCRIPT RUN
}
tclquit
\end{minted}

%
% END SWITCH CONFIG
%




%
% END DOCUMENT
%
\end{document}
